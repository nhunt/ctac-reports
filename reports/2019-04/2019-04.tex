\documentclass[nonacm,acmtog]{acmart}
\usepackage{xcolor}
\usepackage{graphicx}
\usepackage{url}
\usepackage{subcaption}
\usepackage{hyperref}
\newcommand{\todo}[1]{\textcolor{red}{\textbf{TODO}: #1}}
\newcommand{\TODO}{\textbf{\textcolor{red}{TODO}}}
\newcommand{\secref}[1]{\S\ref{#1}}


% Based off of Google Drive "Draft 1"

% ==============================================================================
% ==[ Header information ]======================================================
% ==============================================================================

\title{CTAC-2019-04: Lightweight Climbing Harnesses}
\subtitle{Climbing Technical Advisory Committee}

% ==============================================================================
% ==[ Abstract ]================================================================
% ==============================================================================

\begin{abstract}
  This report discusses the use of lightweight climbing harnesses within the
  context of Mountaineers Basic Climbing classes and experience climbs.
  Lightweight equipment is popular for good reason, however CTAC recommends
  that lightweight harnesses are not appropriate in beginning courses where a
  variety of techniques are introduced. In addition, CTAC recommends that
  lightweight harnesses not be used on experience climbs. On glacier climbs,
  lightweight harnesses may be used if specifically allowed by the climb
  leader.
\end{abstract}

\begin{document}
\maketitle

% ==============================================================================
% ==[ Introduction ]============================================================
% ==============================================================================

\section{Introduction}
\label{sec:intro}
  
  In June, 2019, the Mountaineers Safety Committee requested a CTAC
  investigation of the use of lightweight harnesses in Basic Climbing classes
  and experience climbs. This request was prompted by an incident report
  (5/29/2019). In this incident, a student was practicing rappelling on rock
  using a ski-mountaineering harness (Black Diamond Couloir). Approximately
  halfway through the rappel, the student reported that the leg loop on the
  harness had ``snapped''. This was not a catastrophic harness failure, and the
  student was on a backup belay, so there was no accident or injury resulting
  from the failure. However, it was alarming for both the student and
  instructors.

% ==============================================================================
% ==============================================================================

\section{Discussion}
\label{sec:discuss}
  
  We reviewed the incident report and discussed our experiences of teaching
  harness use in Basic Climbing classes. Primary concerns were student safety
  during activities that place high strain on harnesses, such as rappelling,
  prusiking, and falling. It was also noted that hanging in a lightweight
  harness may cause circulatory damage. We discussed whether lightweight
  harnesses would be appropriate during snow field trips and glacier climbs.

% ==============================================================================
% ==============================================================================

\section{Conclusions}
\label{sec:conclusions}

  The Mountaineers Basic Climbing course is designed to introduce new climbers
  to a wide variety of rock, glacier and snow climbing techniques. These skills
  include belaying, rappelling, and prusiking. Most Basic Climbing students
  will purchase a single harness to be used on all climbs.

  There are a wide variety of harnesses on the market, designed for many types
  of climbing. Publicly available information on harnesses, (e.g., company
  marketing, web reviews and bulletin board comments) often emphasize the
  benefits of weight reduction rather than general-purpose use. Without
  climbing experience, it is difficult for Basic Climbing students to sift this
  information and make an appropriate buying decision. CTAC recommends that a
  rock harness be the standard for Basic Climbing students, as rock harnesses
  are robust, comfortable, and durable. We feel that Basic Climbing instructors
  and mentors should be clear when making recommendations on this critical
  component of the safety system.

  A harness used in Basic Climbing should have the following features:
  \begin{itemize}
  \item Belay loop
  \item Comfortable padded belt
  \item Visually verifiable safety buckle
  \end{itemize}

% ==============================================================================
% ==[ Bibliography ]============================================================
% ==============================================================================

\nocite{bmc:harness,mec:harness,rei:harness,rockandice:harness,en:harness,uiaa:harness,astm:harness}

\bibliographystyle{plain}
\bibliography{../ctac.bib}
\end{document}

% ==============================================================================
% ==[ Workspace ]===============================================================
% ==============================================================================

