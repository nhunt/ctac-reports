\documentclass[nonacm,acmtog]{acmart}
\usepackage{xcolor}
\usepackage{graphicx}
\usepackage{subcaption}
\newcommand{\todo}[1]{\textcolor{red}{\textbf{TODO}: #1}}
\newcommand{\TODO}{\textbf{\textcolor{red}{TODO}}}
\newcommand{\secref}[1]{\S\ref{#1}}


% Based off of Google Drive Draft 1

% ==[ Header information ]======================================================

\title{Fixed-Point Belays}
\subtitle{CTAC-2019-05, Climbing Technical Advisory Committee}

% ==[ Abstract ]================================================================

\begin{abstract}
A fixed point belay is a belay of the leader directly off the anchor master
point, instead of the harness, that transfers the force of a fall onto the
anchor, not the belayer. In recent years this method has become increasingly
popular, especially among guides, as a method to protect leaders against being
dropped by a belayer during a fall. In this advisory note we review the
technique, noting safety considerations and practical concerns along the way.

We find that the technique is impractical for use by Basic-level belayers. It
may have limited use for more experienced Intermediate-level climbers on
multi-pitch routes with bolted anchors and on ice climbs, provided the climbers
are well- practiced with the technique. Although some sources have suggested
using plate and tube type belay devices with a fixed point belay, or using
assisted braking devices (e.g., Gri Gri) with the technique, we find that the
simplest and safest way to use it is with a Munter hitch on the master point
locking carabiner, a recommendation that is generally consistent with the
available literature.

References at the end of this note provide the reader with a wealth of
information, and videos, about how to safely rig and use a fixed point belay.
\end{abstract}

\begin{document}
\maketitle

\section{Introduction}
Being dropped by a belayer is a perennial concern for lead climbers, especially
professionals who guide persons withwho have uncertain skills and/or
experience. Moreover, there are situations that create heightened risk for a
leader despite the skill and experience of the belayer, including:; matters
such as large weight differences between the leader and belayer, a roof above
the belayer’s head, difficult climbing, wet/chossy conditions that raise the
risk of a leader falling before the first piece of pro can be placed, or a
poor/awkward belay stance.

Several methods have been proposed to deal with the above hazards. Some wryly
insist that the leader simply must not fall. Others suggest using a Gri Gri or
other assisted braking device off the harness. Still others may tie their
belayer into the belay anchor, if possible, with a low multi-directional piece
that opposes an upward force. One of the more innovative solutions to the
potential “dropped leader” problem is the fixed point belay, also known as the
aka direct belay oraka “banshee” belay.

A fixed point belay is a belay of the leader directly off the anchor master
point, instead of the harness, that transfers the force of a fall onto the
anchor, not the belayer. There are several ways to do this. One of the first
comprehensive descriptions of the technique was presented by IFMGA Guide Chris
Semmel to a meeting of international guides held in Piani Resinelli, Italy in
2005. In his technical report of that meeting American guide, Mark Houston,
described Semmel’s technique and Semmel’s test findings, which demonstrated the
feasibility of the technique.  Not long after that meeting the fixed point
belay attracted the interest of the Association of Canadian Mountain Guides
(ACMG) and the American Mountain Guide Association (AMGA). Meanwhile, the
technique had already gained popularity and acceptance among some Swiss guides,
as reported by ENSA Chamonix (see video in the references).

No attempt will be made in this note to illustrate the fixed point belay. To do
so would be inordinately complex, potentially confusing and possibly even
misleading. Like most climbing related technical matters, it is often easier
for the reader to understand it by viewing a demonstration or video.  The
reader is advised to stop reading and view the ACMG video below before
proceeding further.

https://vimeo.com/44869774

What follows assumes that the reader has watched the video and now has a basic
understanding of the technique.

In reviewing the fixed point belay CTAC consulted the available literature and
watched the available videos on the technique and then conducted their own
informal examination of it at the Seattle Program Center in August 2019. We
posed a number of questions:

\begin{enumerate}
\item Is the technique, when properly setup and executed, reasonably safe?
\item How hard is it to learn? Could we expect basic and intermediate climbers
to learn it in a reasonably short period of time? Would it be worth their
effort to learn it?
\item Is the technique efficient in the sense of saving time compared to more
conventional techniques?
\item How often will a climber encounter situations where the benefits of the
technique outweigh its disadvantages? What are those situations?
\item What are the risks associated with using the technique? What are the
risks of not using it when circumstances might otherwise commend its use?
\item Are some variations of the technique recommended over others?
\item What has CTAC learned in their “hands on” exploration of the technique?
\end{enumerate}

\section{Is the fixed point belay reasonably safe?}
When rigged correctly (i.e., on two reliable pieces of protection that can
sustain both upward as well as downward force, such as bolts or well-placed ice
screws), tThe simple answer is “‘yes”’. if it is rigged correctly on two
reliable pieces of pro that can sustain a high upward as well as downward force
(e.g., bolts, bomber ice screws). As shown in the video there are two basic
ways to rig it:. tThe first uses a Munter hitch on a locking carabiner., and
tThe second uses a plate or tube- style belay device. With the latter method
the brake strand must be redirected to a higher locking carabiner to brake a
fall should the fall occur before the climber places their first piece of
protection. Once a piece of protection has been placed this piece of pro
substitutes for the redirect carabiner, which can then be removed. However,
should the first piece of pro fail before a second piece is placed, or should
several pieces zipper out, this could lead to catastrophic anchor failure. This
is never a possibility if using a Munter hitch instead of a belay device. For
this reason CTAC recommends that climbers avoid using a tube or plate style
belay device with a fixed point belay.

\section{Assisted Braking Devices}
Assisted braking devices (e.g. Gri Gri) were briefly addressed by Chris Semmel
in his original report of the fixed point belay. While Semmel reported that a
Gri Gri was “generally adequate” for use with a fixed point belay he ultimately
recommended that a belay device that allowed some dynamic slippage while
catching a fall was the preferred type of belay device. In passing, we note
that Climbing Technology (CT) has advocated the use of their Alpine Up (in
assisted belay mode) with a fixed point belay. Aside from demonstrating the
technique toward the end of their promotional video, CT appears not to have
reported field test data on this use of their belay device.

The Gri Gri poses some potentially  serious problems for use with a fixed point
belay. First, because there is little or no slippage of rope through the device
when catching a fall, using it in a fixed point belay will cause a higher
impact force on the system, particularly at the top most anchor. Second, it can
be difficult to safely pay out rope to the leader in the manner recommended by
Petzl (Gri Gri manufacturer) when rigged as a fixed point belay. Third, when
rigged as a fixed point belay there is a danger that during a fall the Gri Gri
could contact the rock in manner that defeats the camming mechanism, thus
causing the device to fail. For these reasons we recommend against using a Gri
Gri directly with a fixed point belay. The degree to which these same concerns
apply to the Climbing Technology Alpine Up is unknown, although it seems
obvious that due to its static catch of a fall it too will impart higher impact
forces on the system. Until more field testing has been reported with the
Alpine Up, we urge caution in using it with a fixed point belay \footnote{One
recently developed use of the fixed point belay with an assisted braking device
(ABD) has been described by AMGA Technical Director and IFMGA Guide, Dale
Remsberg. Dubbed the hybrid system, it initially employs a conventional fixed
point belay using a Munter hitch or redirected plate until the leader places
reliable gear. After that, the belayer uses an ABD, e.g., a Gri Gri, to belay
the leader.

Prior to starting the pitch the leader pre-rigs the belayer with a Gri Gri on
his or her harness belay loop on slack rope about as far below the fixed point
as the distance to the first piece of reliable gear. When the leader reaches
and clips that piece of gear the system can be easily and safely transitioned
from the fixed point belay to the ABD. Taking in any excess slack and keeping a
hand on the brake strand below the ABD, the belayer then disassembles the fixed
point belay.  From that point forward the belay is on the Gri Gri on the
belayer’s harness belay loop.

The advantage of the hybrid system is that it balances the risk of the belayer
losing control of the belay after possibly being slammed into the wall during a
leader fall on a pitch with difficult climbing before the first piece can be
placed against potential rockfall later on that pitch that could disable the
belayer, causing loss of control of the fixed point belay. When to use the
hybrid fixed point system is a complex risk management judgement call that
relies heavily on leader experience, knowledge and balancing of the objective
risks posed by the pitch, and sufficient training of the belayer to safely
execute the procedure.}.

\section{Teaching the fixed point belay}
The fixed point belay was developed as a technique to protect lead climbers, in
particular guides,  who are sometimes belayed by persons whose belay skills are
uncertain. Guides who advocate the use of the fixed point belay believe that
when necessary their guests can be taught to use this belay technique, but it
is unclear how much time is required to safely teach the technique to those
inexperienced with it. Presumably, if a guide anticipates using a fixed point
belay, they will spend time teaching the method to their guest before the
climb.

During examination of the fixed point belay CTAC discovered that paying out
rope to the lead climber can be awkward and taxing. We found no good way to
mitigate this inconvenience, which means that leaders who use it must carefully
weigh the need for the technique against the time needed to teach it to a
belayer and the awkwardness the belayer may experience while using it.
Therefore, CTAC recommends against teaching this technique to Basic-level
climbers. CTAC recognizes that there may be occasions when Intermediate-level
climbers might benefit from using the technique, but it is imperative that
climbing partners who use it are well practiced and experienced with it before
using it on a climb.

\section{When to consider using the fixed point belay}
As noted in the ACMG article and other sources (see references), a fixed point
belay is best considered an option in the following situations:

\begin{itemize}
\item Large weight differences between a belayer and a climber
\item High potential for a leader fall (difficult grade, crux move off the
belay, wet rock)
\item Potential for high impact force (high fall factor)
\item Potential for a long leader fall (slab routes, long run-outs, or ice
routes)
\item Problems maintaining the integrity of the belay (inexperienced belayer,
icy ropes, poor stance, roof above the belayers head, belay cave)
\end{itemize}

An important question to ask before using a fixed point belay is whether a
simpler method might instead be used to prevent the belayer from a hard upward
force during a leader fall. In some cases a multidirectional piece that can
sustain a high upward force (e.g., a bomber cam) can be placed below the anchor
masterpoint and the belayer clipped in to it, thus limiting the distance that
the belayer can be pulled upward and inward during a fall. However, there is an
important trade-off: The belayer will not be able to execute as dynamic a catch
of the fall as might otherwise be possible, which will subject the system
(especially highest piece of placed pro) to higher fall forces. In passing, we
note that fixed point belays may be preferable precisely because there is no
opportunity to add a reliable lower oppositional piece to the anchor, as is the
case for many two-bolt belay anchors.

\section{Safety considerations and tips}
The ACMG article and related video (see reference below) is great source for
learning to rig a fixed point belay.

\begin{enumerate}
\item First and foremost, fixed point belay anchors must be bomber and capable
of handling a high upward force as well as a downward force. Two pieces of pro
are required. Pro may be arranged either vertically or horizontally, the latter
of which is the common orientation for bolted anchors. Although some European
sources have suggested using well-placed cams or pitons, most sources recommend
using bolts or ice screws. Vertical orientation is preferred for ice screws
based on the well known proclivity for ice to fracture horizontally.

\item  Fixed point anchors are not equalized, but instead rely on redundancy
for safety back-up, which further emphasizes the need for both pieces of pro to
be bomber.

\item The master point needs to be placed at chest level. Higher or lower
placement makes the anchor particularly awkward to use when paying rope out to
the leader. This is one reason why we do not recommend assisted belay devices
for use with a fixed point anchor.

\item See ACMG illustrations on pages 4-6.  A locking master carabiner (holding
the Munter hitch) needs to be clipped to a small loop to a piece of pro
(preferably a bolt or ice screw) so that upward movement of the carabiner
during a fall is no more than 20cm (less than 8 inches). In the event of a fall
this reduces shock loading on the pro and also keeps the Munter hitch within
reach of the belayer. As seen in the ACMG illustrations there are several ways
to tie the anchor runner and tie into the anchor. All rigging methods create a
doubled loop to hold the master point carabiner to enhance anchor strength.
Runners should be made from high strength cord (7mm nylon accessory cord or
equivalent). When using ice screws ACMG recommends that the master carabiner be
clipped through both the shelf and small master point loop (see the diagram on
page  6 of their article). This limits shock loading on vertical configurations
of the system.

\item In their exploration of the fixed point belay, The American Alpine
Institute (AAI) discovered that the belayer’s brake hand should be kept well
away from the master carabiner. This permits better braking control and
prevents the belayer from being pulled into the wall.

\item  A tube or ATC type belay device can be used in place of the Munter
hitch, but the brake strand must be redirected until the climber places his or
her first piece of pro. However, if this first piece should fail before a
second piece is placed, or should the protection system zipper, the belay
device will fail to stop the fall.  For this reason, a Munter belay is
preferred over a belay that uses an ATC or tube type device.

\item The master carabiner should not be clipped directly to a bolt hanger (or
ice screw hanger) as this can create dangerous metal-to-metal twisting forces
that could break a carabiner or hanger during a fall.  Welded Fixe rings are
considered safe due to their strength and ability to adjust to dynamically
changing impact forces.

\item It is recommended that the belayer wear gloves as fixed point belays are
most likely to be used in situations where the belayer may need to catch a high
impact fall.
\end{enumerate}

\section{Who is best suited to use a fixed point belay?}
The fixed point belay was developed by professional guides for use by guides
and their guests in situations where a guest may not be able to catch and hold
a guide fall. A key consideration is whether the need for the technique
sufficiently offsets the time taken to train a belayer to effectively use it.
Most recreational climbers will likely find limited applications for a fixed
point belay. It is doubtful that it would be worth teaching the technique to
Basic-level climbers as Basic climbs are generally low fifth class climbs where
leader falls are unlikely, catching a leader fall is rarely an issue and
opportunities to place a low multi-directional piece to prevent a strong upward
pull on the belayer are often available. Intermediate-level climbers may
occasionally find use for a fixed point belay on hard multi-pitch routes with
bolted belay anchors and waterfall ice routes but it is imperative that
climbing teams are well practiced in its use and well aware of its limitations
and safety considerations.

\section{Potential problems not addressed by current sources}
None of the sources listed in the references below address what a belayer using
a fixed point belay should do after catching a leader fall, particularly should
the leader be injured or require rescue. The belayer will need to improvise an
effective accident response based on their current knowledge and skill. Current
leader tie-off methods taught to Mountaineers Basic Course students do not
address how to perform a leader tie-off when the anchor master point is in
front, vis a vis behind, the belayer, let alone how to perform the procedure
when belaying off the anchor instead of the harness.
\end{document}

\begin{comment}
Endnotes:
References:
Chauvin, M. & Coppolillo, R. 2017. The mountain guide manual: The comprehensive                        reference-from belaying to rope systems and self rescue. Guilford, Connecticut: Falcon Guides. Pp 68-70.
ACMG video:
https://vimeo.com/44869774
American Alpine Institute blog:
http://blog.alpineinstitute.com/2017/11/fixed-point-belay-techniques.html
ENSA Chamonix video:
https://m.youtube.com/watch?time_continue=3&v=eqZQnCGl24A
ACMG article:
https://www.acmg.ca/05pdf/TechFiles/TechFiles_Vol1_No1.pdf
Mike Barter video:
https://vimeo.com/45225097
Andy Kirkpatrick “Banshee Belay”, (see Chapter 6, section 11):
http://multipitchclimbing.com
Mark Houston 2005 paper (IFMGA)
https://files.meetup.com/1778479/IFMGA_on_lead_belays_5_2005.pdf
Alpine UP video (last part demos a fixed point belay with the device)
https://m.youtube.com/watch?v=EogWFgH_5kE&time_continue=216
\end{comment}
