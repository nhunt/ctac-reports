\documentclass[nonacm,acmtog,authordraft]{acmart}
\usepackage{xcolor}
\usepackage{graphicx}
\usepackage{subcaption}
\newcommand{\todo}[1]{\textcolor{red}{\textbf{TODO}: #1}}
\newcommand{\TODO}{\textbf{\textcolor{red}{TODO}}}


% Based off of Google Drive "Draft 1"

% ==============================================================================
% ==[ Header information ]======================================================
% ==============================================================================

\title{CTAC-2019-03: Pre-Rigged Rappels}
\subtitle{Climbing Technical Advisory Committee}

% ==============================================================================
% ==[ Abstract ]================================================================
% ==============================================================================

\begin{abstract}
  Pre-rigged rappels are where the climbing team's rappel devices are all
  rigged for rappel before the first rappeller descends. The traditional
  method---not pre-rigging rappels---is where each climber attaches their
  rappel device to the rope, descends, and at the next anchor or on the ground,
  removes their rappel device before the next climber attaches their rappel
  device.

  In this report, CTAC recommends use of pre-rigged rappels for both
  Mountaineers field trips and climbing classes.  It describes not only the
  benefits but also the drawbacks of pre-rigged rappels in terms of safety,
  ease of use, and conditions under which is safe to use and finally, those
  where it should not be used.
\end{abstract}

\begin{document}
\maketitle

% ==============================================================================
\section{Introduction}
% ==============================================================================

  Pre-rigged rappels are widely used in the guiding community which enables a
  professional guide to setup the rappel for their novice client.  It ensures
  the safety of the novice who may be uncomfortable in high angle terrain and
  unable to rig their rappel with confidence.  In a multi-pitch setting,
  pre-rigging also allows the guide to descent first, prepare for the next
  rappel, and give their client a fireman's belay.

  For private parties, the general acceptance of pre-rigging rappels is
  difficult to gauge although likely it is not commonly used.  This may be due
  to the lower publicity than other rappelling safety techniques such as using
  a rappel backup, tying knots in the end of the ropes, etc.  For example, the
  AAC article Rappelling published March 2018, mentions pre-rigging but only in
  passing.

% ==============================================================================
\section{Setting up a Pre-Rigged Rappel}
% ==============================================================================

  Prepare each climbers' rappel slings for an extended rappel.  See CTAC report
  Personal Tethers and the Extended Rappel.  Attach each climber to the rappel
  anchor with their personal tether, Attach the climber's rappel device to the
  rope with the first person to rappel at the lowest point on the rope.  Attach
  an autoblock for the first person to rappel.  The other climber's rappel can
  be backed up with an autoblock (preferred) or with a fireman's belay by the
  first rappeller.  Each climber checks the other's climber's rappel setup.
  Before rappelling, each climber tests their rappel setup by weighting the
  rope with the rappel device with slack in the personal tether The first
  rappeller detaches their personal tether and rappels to the ground or the
  next anchor When the first rappeller goes off rappel, the second rappeller
  detaches their personal tether and rappels.

% ==============================================================================
\section{Benefits to Pre-rigging a rappel}
% ==============================================================================

  Pre-rigging the rappels has several safety and efficiency benefits:

  \begin{enumerate}
  \item Allows each climber to double check each other's rappel setup.  Which
    double checking is always suggested, it is especially important at the end
    of a long day when climbers are tired or when conditions are difficult,
    such as when it is dark, cold, wet, windy, etc.
  \item Allows an experienced climber help a novice setup their rappel rather
    than leaving the novice to fend for themselves in a stressful and
    distracting environment.
  \item Improves overall speed of everyone getting down the rope ``.. because
    it eliminates the downtime of waiting for climbers to rig the rappel one by
    one.  After rigging up, everyone can be on the tensioned rope without being
    yanked around because of the extension. When Rapper 1 is on rappel, Rapper
    2 is putting on their autoblock knot. The moment the Rapper 1 goes off
    rappel, they quickly feed a couple meters of rope through their device and
    Rapper 2 can head down immediately. The movement of climbers down the rope
    is pretty much constant, with no waiting for someone to rig.''-
    alpinesavvy.
  \item When quads are used for the safety system at the rappel chains---a
    safety and efficiency technique useful in a multi-rappel setting---the last
    climber to rappel can remove the quad and store it while the first
    climber(s) are rappelling.
  \item If the first rappeller is tied into the rope, the other end does not
    need a knot tied in. This is because the second rappeller keeps the rope
    locked off.  Not having a knot is one strand is one less thing to worry
    about potentially getting stuck. Also when rappelling on a clean wall and
    rope being pulled from the anchor above sails past the climbers, it is
    unnecessary to retrieve the end to tie a knot in the end.
  \item If three people are rappelling, the pre-rigged rappel allows two
    climbers to rappel on a single strand at the same time without incurring
    the risks of counter balanced rappels (colloquially known as a
    simul-rappel).  The rappel device of the last rappeller locks the rope
    allowing the first two rappellers to weight their strand independently.
  \end{enumerate}

% ==============================================================================
\section{Downsides to Pre-rigging a rappel}
% ==============================================================================

  There are disadvantages to pre-rigging and times when it could not be used.
  These include:

  \begin{enumerate}
  \item When crowding of more than three people make it impractical or even
    dangerous to pre-rig everyone.  However, the first two or three
    climbers---depending on the space around the anchor---can still pre-rig.
  \item The first person down cannot do a pull test.  If there is a risk that
    the rope will not pull cleanly, it is better to not pre-rig and have the
    first person down do a pull test than to pre-rig and discover the rope
    cannot be retrieved.
  \item If the first person down were to become incompacitated such as by rock
    fall, the second rappeller could not exit the system without cutting their
    tether---another reason to always carry a small knife. 
  \item Pre-rigging should not be used in situations in which the rappel starts
    above the rappel anchor (e.g. Ingall Peak South Ridge).
  \end{enumerate}


% ==============================================================================
\section{System Complexity and Ease of Visual Assessment}
% ==============================================================================

  Multiple rappel devices attached to the rappel rope adds complexity that
  could make visual assessment of the rigging more difficult especially in an
  adverse environment such as low lighting, wind, and rain.  However, this is a
  minor consideration with respect to the benefits of each climber cross
  checking the other climbers' rigging as well and their own.

% ==============================================================================
\section{Ability to Test Before Use}
% ==============================================================================

  Pre-rigging the rappel does not change the ability to test the rappel before
  use.  That is, with the personal tether attached and with slack, the climber
  next up to rappel can test if the rappel device has been properly rigged
  before removing the tether.  It is conceivable, however, that the act of
  cross checking or assumed cross checking could engender complacency in
  testing.  For this reason the CTAC recommendation that when using pre-rigged
  rappels:

  \begin{enumerate}
  \item Cross checking is explicit with verbal confirmation.
  \item Rappel rigging is tested with the personal tether attached and slack
    before the tether is removed.
  \end{enumerate}

% ==============================================================================
\section{Branch Acceptance and Recommendations}
% ==============================================================================

  The Mountaineers' climbing classes teach pre-rigged rappels as follows:

  \begin{table}
  \begin{tabular}{rll}
    Branch     & Climbing Class & Taught \\
    Everett    & Basic & No \\
               & Intermediate (LOR) & Considering \\
    Bellingham & Basic & Demonstrated \\
               & Intermediate & Yes \\
    Kitsap & --- & No \\
    Olympia & --- & No \\
    Seattle & Basic & No \\
            & Intermediate (rock, ice) & No \\
            & Advanced Alpine Rock     & Yes \\
    Tacoma  & --  & No
  \end{tabular}
  \caption{}
  \end{table}

% ==============================================================================
\section{CTAC Endorsement and Recommendation}
% ==============================================================================

  CTAC endorses the use of pre-rigged rappels whenever safe to do and when the
  conditions described in Downsides to Pre-rigging a rappel are not present.
  Furthermore, the CTAC majority opinion---with some dissenting opinions ---is
  that pre-rigging should be the standard operating procedure taught by The
  Mountaineers' climbing classes used on Mountaineers' field trips again under
  safe conditions.

% ==============================================================================
\section{Conclusions}
% ==============================================================================

  Pre-rigging rappel devices have distinct safety and efficiency benefits
  compel its use not only in the professional guiding situations but also for
  the private and club climbing community. It is especially useful for
  multi-pitch climbs where there are multiple rappels and when there is a mix
  of strong and weak climbers on the team.  However, it should not be used
  blindly for all rappelling situations as there are important downsides and
  hazards that must be considered.

% ==============================================================================
\section{References}
% ==============================================================================
   
   \begin{itemize}
   \item Pre-rigged rappel  is discussed in Mountain Project in 2017 
   \item Recommended use in Rock \& Ice Ask the Masters in 2016,
   \item Benefits described in 2015 article in Climbing by Dale Remsberg.
   \item Benefits and downsides identified in Alpine Savvy blog.
   \item The Mountain Guide Manual, Marc Chauvin and Rob Coppolillo, ppg
     182-183
   \item AMGA reference for guides
   \end{itemize}

%\bibliographystyle{plain}
%\bibliography{../ctac.bib}
\end{document}

% ==============================================================================
% ==[ Workspace ]===============================================================
% ==============================================================================

