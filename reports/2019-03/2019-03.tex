\documentclass[nonacm,acmtog]{acmart}
\usepackage{xcolor}
\usepackage{graphicx}
\usepackage{subcaption}
\newcommand{\todo}[1]{\textcolor{red}{\textbf{TODO}: #1}}
\newcommand{\TODO}{\textbf{\textcolor{red}{TODO}}}
\newcommand{\secref}[1]{\S\ref{#1}}


% Based off of Google Drive "Draft 1"

% ==============================================================================
% ==[ Header information ]======================================================
% ==============================================================================

\title{CTAC-2019-03: Pre-Rigged Rappels}
\subtitle{Climbing Technical Advisory Committee}
\author{Jeff Hunt}

% ==============================================================================
% ==[ Abstract ]================================================================
% ==============================================================================

\begin{abstract}
  Pre-rigged rappels are where the climbing team's rappel devices are all
  rigged for rappel before the first rappeller descends. The traditional
  method---not pre-rigging rappels---is where each climber attaches their
  rappel device to the rope, descends, and at the next anchor or on the ground,
  removes their rappel device before the next climber attaches their rappel
  device.

  In this report, CTAC recommends use of pre-rigged rappels for both
  Mountaineers field trips and climbing classes.  It describes not only the
  benefits but also the drawbacks of pre-rigged rappels in terms of safety,
  ease of use, and conditions under which is safe to use and finally, those
  where it should not be used.
\end{abstract}

\begin{document}
\maketitle

% ==============================================================================
\section{Introduction}
\label{sec:intro}
% ==============================================================================

  Pre-rigging a rappel is a technique that enables multiple climbers to
  simultaneously set up and check their rappels.  It has the potential to
  increase both climber safety and speed, by enabling partner checks and
  parallelizing the rigging.  In scenarios where an experienced climber is
  descending with an inexperienced climber, it allows the more experienced
  climber to set up the rappel for their novice, who may be uncomfortable in
  high angle terrain or unable to rig their own rappel with confidence.

  For private parties, the general acceptance of pre-rigging rappels is
  difficult to gauge although likely it is not commonly used.  This may be due
  to the lower publicity than other rappelling safety techniques such as using
  a rappel backup, tying knots in the end of the ropes, etc.  For example, the
  AAC article Rappelling published March 2018, mentions pre-rigging but only in
  passing.

  This report begins by describing how to properly set up a pre-rigged rappel
  in \secref{sec:setup}. This is followed by an in-depth discussion of the
  benefits of this technique in \secref{sec:benefits}, followed by a discussion
  of the weaknesses and trade-offs in \secref{sec:limitations}.

% ==============================================================================
\section{Setting up a Pre-Rigged Rappel}
\label{sec:setup}
% ==============================================================================

  A detailed visual walkthrough of setting up the pre-rigged rappel is
  available from the Seattle Mountaineers~\cite[time
  4:55]{mountaineers:rappel-video}, though we briefly describe the process
  below.

  First, prepare each climber's rappel sling for an extended rappel, as
  described in CTAC 2019-02~\cite{ctac:2019-2}, ensuring each climber is
  attached to the rappel anchor with their personal tether.  Then each climber
  can independently attach their rappel device to the rope, ensuring the device
  of the first person to rappel is at the lowest point on the rope.  Finally,
  the first person to rappel attaches an autoblock backup to the rope.

  The other climber's rappel can be backed up with an autoblock (preferred) or
  with a fireman's belay by the first rappeller.  Each climber checks the
  other's climber's rappel setup.  Before rappelling, each climber tests their
  rappel setup by weighting the rope with the rappel device with slack in the
  personal tether The first rappeller detaches their personal tether and
  rappels to the ground or the next anchor When the first rappeller goes off
  rappel, the second rappeller detaches their personal tether and rappels.

\subsection{Testing Before Use}

  Multiple rappel devices attached to the rappel rope adds complexity that
  could make visual assessment of the rigging more difficult especially in an
  adverse environment such as low lighting, wind, and rain.  However, this is a
  minor consideration with respect to the benefits of each climber cross
  checking the other climbers' rigging as well and their own.

  Pre-rigging the rappel does not change the ability to test the rappel before
  use.  That is, with the personal tether attached and with slack, the climber
  next up to rappel can test if the rappel device has been properly rigged
  before removing the tether.  It is conceivable, however, that the act of
  cross checking or assumed cross checking could engender complacency in
  testing.  For this reason the CTAC recommendation that when using pre-rigged
  rappels:

  \begin{enumerate}
  \item Cross checking is explicit with verbal confirmation.
  \item Rappel rigging is tested with the personal tether attached and slack
    before the tether is removed.
  \end{enumerate}


% ==============================================================================
\section{Benefits to Pre-rigging a rappel}
\label{sec:benefits}
% ==============================================================================

  Pre-rigging rappels has several benefits, broadly classified as either
  improving climber safety or group efficiency.

\subsection{Increased Climber Safety}

  Pre-rigging allows each climber's rappel setup to be checked by other members
  of their group.  This is true even for the last person to descend, who
  traditionally does not have the opportunity for a partner check.

  Additionally, as described briefly in \secref{sec:intro}, pre-rigged rappels
  allow experienced climbers to help a novice setup their rappel, rather than
  leaving the novice to fend for themselves in a stressful and distracting
  environment.  The more experienced climber can then descend first, find and
  prepare the next rappel station, and provide a fireman's belay to the
  novice from below.  The weight of the experienced climber on the rope will
  prevent the novice climber from descending early, and since the experienced
  climber rigged and checked the novice's rappel before descending, there's no
  concern over the novice mis-rigging their rappel.

\subsection{Increased Group Speed}

  In addition to improved safety, pre-rigged rappels can greatly enhance a
  climbing group's efficiency.  A pre-rigged rappel can improve the overall
  speed of everyone getting down the rope i

  This is described in detail in an Alpine Savvy blog posting as
  follows\cite{alpinesavvy:pre-rigged-rappels}:

  \begin{quote}
  ``... it eliminates the downtime of waiting for climbers to rig the rappel
  one-by-one.  After rigging up, everyone can be on the tensioned rope without
  being yanked around because of the extension. When Rapper 1 is on rappel,
  Rapper 2 is putting on their autoblock knot. The moment the Rapper 1 goes off
  rappel, they quickly feed a couple meters of rope through their device and
  Rapper 2 can head down immediately. The movement of climbers down the rope is
  pretty much constant, with no waiting for someone to rig.''
  \end{quote}

  Another efficiency gain comes when quads are used as the safety system at the
  rappel chains---a technique that's useful in a multi-rappel setting.  Since
  the final rappeller will be secured to the climbing rope with their rappel
  device, they can tear down the quad while the second-to-laste climber is
  descending.  This way they will be ready to descend as soon as the previous
  climber reaches the next anchor.

  %  If the first rappeller is tied into the rope, the other end does not
  %  need a knot tied in. This is because the second rappeller keeps the rope
  %  locked off.  Not having a knot is one strand is one less thing to worry
  %  about potentially getting stuck. Also when rappelling on a clean wall and
  %  rope being pulled from the anchor above sails past the climbers, it is
  %  unnecessary to retrieve the end to tie a knot in the end.

  Finally, a slightly more advanced use of the pre-rigged rappel allows for two
  climbers to safely rappel simultaneously---each climber using a single strand
  of the rope---as long as there is at least one climber remaining at the top
  of the pitch.  The pre-rigged rappel of the remaining climber locks the rope
  in-place, avoiding many of the risks associated with counter-balanced rappels
  and allowing each rappeller to weight their own strand independently.
  \todo{This could use a reference.}

% ==============================================================================
\section{Limitations of Pre-Rigged Rappels}
\label{sec:limitations}
% ==============================================================================

  Although pre-rigged rappels provide a number of benefits, there are also some
  disadvantages that should be taken in to account when deciding on a rappel
  strategy.  Additionally, there are some environmental considerations that may
  make pre-rigged rappels an inappropriate choice strategy to employ.

\subsection{Disadvantages to Pre-Rigged Rappels}
  \begin{enumerate}
  \item The first person down cannot do a pull test.  If there is a risk that
    the rope will not pull cleanly, it is better to not pre-rig and have the
    first person down do a pull test than to pre-rig and discover the rope
    cannot be retrieved.
  \item If the first person down were to become incompacitated such as by rock
    fall, the second rappeller could not exit the system without cutting their
    tether---another reason to always carry a small knife.
  \end{enumerate}

\subsection{Inappropriate Environments}

  \begin{enumerate}
  \item When crowding of more than three people make it impractical or even
    dangerous to pre-rig everyone.  However, the first two or three
    climbers---depending on the space around the anchor---can still pre-rig.
  \item Pre-rigging should not be used in situations in which the rappel starts
    above the rappel anchor (e.g. Ingall Peak South Ridge).
  \end{enumerate}

% ==============================================================================
\section{Branch Acceptance and Recommendations}
% ==============================================================================

  The Mountaineers' climbing classes teach pre-rigged rappels as follows:

  \begin{table}
  \begin{tabular}{|rll|}
    \hline
    \textbf{Branch} & \textbf{Class} & \textbf{Taught?} \\
    \hline\hline
    Everett    & Basic & No \\
               & Intermediate (LOR) & Considering \\
    \hline
    Bellingham & Basic & Demonstrated \\
               & Intermediate & Yes \\
    \hline
    Kitsap & --- & No \\
    \hline
    Olympia & --- & No \\
    \hline
    Seattle & Basic & No \\
            & Intermediate (rock, ice) & No \\
            & Advanced Alpine Rock     & Yes \\
    \hline
    Tacoma  & --  & No \\
    \hline
  \end{tabular}
  \caption{Current curriculum status of pre-rigged rappels in various courses
  at the Mountaineers branches}
  \end{table}

% ==============================================================================
\section{CTAC Endorsement and Recommendation}
% ==============================================================================

  CTAC endorses the use of pre-rigged rappels whenever safe to do and when the
  conditions described in Downsides to Pre-rigging a rappel are not present.
  Furthermore, the CTAC majority opinion---with some dissenting opinions ---is
  that pre-rigging should be the standard operating procedure taught by The
  Mountaineers' climbing classes used on Mountaineers' field trips again under
  safe conditions.

% ==============================================================================
\section{Conclusions}
% ==============================================================================

  Pre-rigging rappel devices have distinct safety and efficiency benefits
  compel its use not only in the professional guiding situations but also for
  the private and club climbing community. It is especially useful for
  multi-pitch climbs where there are multiple rappels and when there is a mix
  of strong and weak climbers on the team.  However, it should not be used
  blindly for all rappelling situations as there are important downsides and
  hazards that must be considered.

% ==============================================================================
\section{References}
% ==============================================================================

   \begin{itemize}
   \item Pre-rigged rappel  is discussed in Mountain Project in 2017
   \item Recommended use in Rock \& Ice Ask the Masters in 2016,
   \item Benefits described in 2015 article in Climbing by Dale Remsberg.
   \item Benefits and downsides identified in Alpine Savvy blog.
   \item The Mountain Guide Manual, Marc Chauvin and Rob Coppolillo, ppg
     182-183
   \item AMGA reference for guides
   \end{itemize}

\bibliographystyle{plain}
\bibliography{../ctac.bib}
\end{document}

% ==============================================================================
% ==[ Workspace ]===============================================================
% ==============================================================================

